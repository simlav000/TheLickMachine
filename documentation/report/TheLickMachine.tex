\documentclass[12pt]{article}
\usepackage{graphicx} % Required for inserting images
 \usepackage{hyperref}
\usepackage[letterpaper, margin=1in]{geometry}

\title{Recognizing \textit{The Lick} - Project Proposal}
\author{Maddy Walkington, Louis Bouchard, Simon Lavoie, Simon Pino-Buisson}
\date{November 2024}

\begin{document}

\maketitle

\begin{center}
    \textbf{COMP 451}
\end{center}
\section*{Vectorizing the Dataset}
Andy Chamberlain and IDK Wein’s \textit{The Lick} dataset was used to train this model. This dataset is composed of 18 thousand unique audio snippets. Roughly half of them contain examples of \textit{The Lick} while the other half contain a different melody. Within these two categories, there are three subcategories. These are:

\begin{itemize}
    \item \textbf{solo}: These are single midi instruments playing in isolation. 
    \item \textbf{combo}: Two or more midi instruments playing in concert. 
    \item \textbf{External}: Music samples from official recordings. 
\end{itemize}
It was found that 98\% of the audio files were under 4 seconds in length, and so each audio file was either padded or trimmed to 4 seconds to ensure consistency in the data. With each audio file, four greyscale images are produced. 

\subsection*{Mel Spectrograms}
The first greyscale image produced is a Mel-Spectrogram. This is done by performing a sequence of Short-Time Fourrier Transforms (STFT) on IDK-HOW-LONG time window divisions of the total 4s sample.
The result is a spectrogram which can be rescaled to the Mel-scale, which puts emphasis on the frequency range of human hearing.
This is done by converting the frequency $f$ into a “mel” $m$ by the formula 
\[ m = 2595\log_{10}\left(1 + \frac{f}{700}\right)\]\cite{SOMETHING}

\subsection{Deltas}
A common method for extracting information about variations in sound over time is through the use of differences in signal features (abbreviated as $\Delta$). Specifically, for a musical feature, such as frequencies plotted in a spectrogram, the $\Delta$ is defined as the change in this frequency between two consecutive time frames. Thus, for some time frame $k$ and some frequency $f_k$, its delta is given by
$$\Delta_k = f_k - f_{k-1}$$.
The second difference, known as the $\Delta\Delta$ is given by
$$ \Delta\Delta_k = \Delta_k - \Delta_{k-1}$$. 
These provide computationally trivial methods of approximating the first and second derivative of a signal, which could be helpful for identifying note changes when trying to recognize \textit{The Lick}.
\bibliographystyle{IEEEtran}

\bibliography{bibliography}
\end{document}
